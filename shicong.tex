\documentclass{aastex62}

\begin{document}

\title{My Title}

\begin{abstract}
We report on a measurement of thermal neutrons, generated by the hadronic component of extensive air showers(EAS),by means of a small array of EN-detectors developed for the PRISMA project(PRImary Spectrum Measurement Array),novel devices based on a compound alloy of ZnS(Ag)and 6LiF. This array has been operated within the ARGO-YBJ experiment at the high altitude Cosmic Ray Observatory in Yangbajing(Tibet,m a.s.l.). 
\end{abstract}


\section{Introduction}

The cosmic ray energy spectrum spans over many decades from about 10^{6}\end eeV to beyond 10^{20}\end eeV.It consists of different regions with power law behavior and changes in the power law index.In the high energy range above 100 TeV two features are known since a long time,that is a steepening of the spectrum,named the knee,at about 3-5\times 10^{15}\end eeV and a hardening,named the ankle,at about 3-5\times 10^{18}\end eeV.Other peculiar features have been observed in this energy interval by the KASCADE-Grande experiment.


\begin{equation}
<n>=36 E^{0.56}
\end{equation}

This is my equation:"$F(x)=\int f(x) {\rm d}x = \int x^2 {\rm d x}=x^3/3$"equation

\begin{equation}
F(x)= \int f(x) {\rm d}x = \int x^2 {\rm d x}=x^3/3
\end{equation}

\section{The EN-detector}
The EN-detector is based on a special phosphor,which is a granulated alloy of inorganic ZnS(Ag) scintillator added with LiF enriched with the isotope up to 90\%.One captures one thermal neutron via the reaction \alpha\end with cross section of 945 barn.The phosphor is deposited in the form of a thin one-grain layer on a white plastic film, which is then laminated on both sides with a thin transparent film.The scintillating compound grains used are of 0.3-0.8 mm in size.The effective thickness of the scintillator layer is 30 mg/cm^{2}\end ..Light yield of the scintillator is ∼160,000 photons per neutron capture.The structure of a typical EN-detector is shown in Fig.1,right. The scintillator of 0.36m^{2}\end aarea is mounted inside a black cylindrical polyethylene (PE) 300-l tank which is used as the detector housing. The scintillator is supported inside the tank to a distance of 36 cm from the photomultiplier (PMT) photocathode.A 6”-PMT (FEU-200) is mounted on the tank lid. A light reflecting cone made of foiled PE foam of 5-mm thickness is used to improve the light collection.As a result,∼100 photoelectrons per neutron capture are collected.The efficiency for thermal neutron detection in our scintillator was found experimentally by neutron absorption in the scintillator layer to be about 20\%.To determine it,we measured the counting rate of our scintillator layer,then we put another similar layer under the first one (with a black paper between them) as an absorber and measured again.Then we compared the results and calculated the scintillator efficiency.Similar efficiency was also obtained by simple Monte-Carlo simulation using GEANT4 code.As an example,we show in Fig.2 the response of the detector illuminated with a low activity source of thermal neutrons (∼1 Bq of 252 Cf).

\begin{eqnarray}
F(x) &=&\int f(x) {\rm d}x \\
&=& \int x^2 {\rm d x} \nonumber\\
&=&x^3 \over 3
\end{eqnarray}


\begin{figure}[ht!]
\plotone{11.eps}
\caption{scan 20190615 evt865 1.}
\end{figure}

\setcounter{table}{0}
\begin{table}[h!]
\renewcommand{\thetable}{\arabic{table}}
\centering
\caption{Detector events} \label{tab:decimal}
\begin{tabular}{D & D}
\tablewidth{0pt}
\hline
\hline
Detector  &  \multicolumn2c{Events} \\
\hline
\decimals
No.33     & 439     \\
No.34     & 1202     \\
No.35     & 897     \\
No.36     & 1038     \\
\hline
\multicolumn{4}{IP11 events of six days}
\end{tabular}
\end{table}


\end{document}
